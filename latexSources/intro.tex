\chapter{Introduction}
Internet was conquered by the client-server-paradigm and by server providers. It started in the 1990th after WWW was invented. Now, in the second decade of the 21th century, nearly any Internet application is made up by a server that provides services to clients. Clients tend to become smaller. Crome OS is an incarnation of that trend. It is a quite basic operating systems that requires Internet access to function.

I don't want to argue about that technical principle. Client-server-paradigm is a very powerful design principle. It has drawbacks, though. Any communication is mediated by a server. It knows at least who communicates with whom. Such data can be analyzed. That's forbidden in a number of countries around a world, especially Europe. Some analyzing activities not only trample personal privacy but are means of steeling intellectual properties. That's a crime.

Centralized systems tend to tempt such activities. Huge server store an incredible number of valuable information. Far more valuable information can be extracted by big data analysis. Who can resist? We make it too easy for criminals. When using cloud services, we trust both - personal integrity of each person that works with the cloud service provider and their professional skills. That trust is naive when it comes to really relevant information as Mr. Snowden revealed least year.

I like options. Sometimes, a client-server-paradigm doesn't fit to the application. Social networks are an example. People want to exchange information. People want to meet others. Social networks are a perfect example for what we call {\it n-to-n communication}. Many talk with many. Client-Server means: One offers services to many what we call {\it 1-to-n communication}.
We call {\it n-to-n communication} also {\it Peer-to-Peer-communication}.

Programmer frameworks dramatically reduce software costs. There are plenty of powerful tools, platforms and even complete programming languages for building Web-based applications. There are a few frameworks for P2P systems. The reason is obvious: Server based applications have usually a business plan. Even it is a bad one, personal data of users can be sold. 

A pure P2P system has no central database that can be target of criminal activities. That's good news for users with concerns about privacy and property. 

Shark is a Java framework that helps to build mobile semantic P2P systems. It is optimized for mobile environment but also works in Internet. Shark applications have a very simple communication principle: Two devices meet and exchange data with TCP, E-Mail, Wifi-Direct etc. That's it. Each devices can remember retrieved data and deliver it again. No server is required - at least you build one. 

It is quite difficult to build a P2P system with Webserver technology. It is quite difficult to build a client-server system with Shark. It is feasible but it hurts. Shark applications are usually pure P2P applications. They can even work without any infrastructure if they are solely based on Wifi-direct.

Shark uses an event-based programming model with a quite elaborated data- and a very simple communication-model. Of course, it is different from client-server paradigm. It might take some time to come along with it. But, it reduces time to create a P2P application dramatically. 

The book describes a basic example in section \ref{sec:knowledgePorts:StandardKP}. Two peers negotiate their interests and exchange data. It takes less than 20 lines of code to produce a fully operational peer-to-peer application. The time you spend with the book is repaid when building applications.

This book is a step by step introduction into the realm of programming pure 
P2P applications with the Shark Framework (SharkFW) release 2. I use each 
chapter as basis for a single lecture. The actual tutorials are taking longer, depending on previous background knowledge. Usually we have weekly lectures and my students are able to write first semantic P2P applications after two months.

Apparently, English isn't my first language. Please, be lenient with me and that book. I apologize for any mistake and bad style. That's the first edition of the book. It isn't finished of course. For me, the hugest problem are redundancies, too lengthy explanations and not to mention the nice pictures ;). I'm working on it. To look on the bright side, each line of code in that book is tested and works fine. Any major function of the framework is explained. I have to confess, I'm glad that we have reached that status in our non-profit open source project.
I want to thank the German ministry for eduction and research for the financial support between 2009 and 2012. I thank my students who worked and work with sometimes incredible enthusiasm. There is no better incentive for a professor. Thanks a lot!

If your are interested in Shark, doing research, want to use Shark in your lectures, planning to set up a project etc. - don't hesitate to contact me\footnote{info@sharksystem.net}.

\vspace{1,5cm}

Thomas Schwotzer, September 2014 in Berlin / Germany
